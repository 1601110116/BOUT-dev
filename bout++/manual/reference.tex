%% Reference manual for C++ BOUT code

\documentclass[12pt, a4paper]{article}
\usepackage[nofoot]{geometry}
\usepackage{graphicx}
\usepackage{fancyhdr}

%% Modify margins
\addtolength{\oddsidemargin}{-.25in}
\addtolength{\evensidemargin}{-.25in}
\addtolength{\textwidth}{0.5in}
\addtolength{\textheight}{0.25in}
%% SET HEADERS AND FOOTERS

\pagestyle{fancy}
\fancyfoot{}
\renewcommand{\sectionmark}[1]{         % Lower case Section marker style
  \markright{\thesection.\ #1}}
\fancyhead[LE,RO]{\bfseries\thepage}    % Page number (boldface) in left on even
                                        % pages and right on odd pages 
\renewcommand{\headrulewidth}{0.3pt}

\newcommand{\code}[1]{\texttt{#1}}
\newcommand{\file}[1]{\texttt{\bf #1}}

\newcommand{\pow}{\ensuremath{\wedge} }
\newcommand{\poweq}{\ensuremath{\wedge =} }

\begin{document}

\title{BOUT++ Reference Manual}
\author{B.Dudson, University of York}

\maketitle

\tableofcontents

\section{Introduction}

This is the reference manual for the BOUT++ code\cite{Dudson2009,dudson-2008-arxiv}, a C++ development of the
BOUT code developed by X.Xu and M.Umansky at LLNL. This document describes
the inner workings of the code. Users of the code who want to modify 
the equations solved should read the users manual first.

\section{Overview}

Classes implemented in BOUT++ are:
\begin{itemize}
\item \code{Field2D} - An object for 2D X-Y profiles. Handles memory allocation transparently,
  and uses overloaded operators to simplify indexing.
\item \code{Field3D} - Wrapper for 3D fields. Again uses overloaded operators.
\item \code{FieldPerp} - Perpendicular (X-Z) fields, primarily used for fields solvers.
\item \code{Vector2D} - An object storing a vector (3 Field2D components) constant in Z.
\item \code{Vector3D} - Stores a 3D vector (3 Field3D components)
\item \code{Solver} - Interface between BOUT++ and the solver (PVODE). All
  functions specific to PVODE are contained in this class.
\item \code{Datafile} - Provides a simple means of reading and writing dump and restart files,
  currently to PDB. Changing the file format means changing this class.
\item \code{Communicator} - Parallel communication object. Wraps up the details
  of transferring data between processors, providing interface independent of MPI or
  details like number of processors and topology.
\item \code{bstencil}
\item \code{bvalue}
\item \code{OptionFile} - reads a settings file, providing an interface to obtain values
  of named variables
\item \code{Output} - A class for writing to stdout and/or a log file.
\end{itemize}

In addition there are several files which provide operations on these objects.
\begin{itemize}
\item \file{grid.cpp} - Reads a uedge.grd file
\item \file{difops.cpp} - provides differential operators
\item \file{boundary.cpp} - Boundary conditions NOTE: NEEDS IMPROVING
\item \file{initialprofiles.cpp} - sets initial profiles for variables
\item \file{invert\_laplace.cpp} - Inverts a laplacian equation using fast linear method
  (FFT + tridiagonal solver)
\item \file{invert\_gmres.cpp} - provides a simple interface for inversion problems,
  intended for inverting for $\phi$ with nonlinear terms. NOTE: NOT FINISHED YET
\item \file{topology.cpp} - calculates the mesh topology
\item \file{utils.cpp} - provides some useful utilities, primarily memory allocation
\item \file{bout++.cpp} - main function
\end{itemize}

The physics model specific code is in \file{2fluid.cpp}, solving
simplified 2-fluid equations. This is intended for benchmarking.

\section{Classes}



\subsection{Field2D, Field3D and FieldPerp}

These field classes describe variables in 2 or 3 dimensions. All meaningful
addition, subtraction, multiplication, division and exponentiation operators
are overloaded, handling indexing transparently. 

This is intended to separate details such as the size of the grid from the policy
of what operations should be performed. This both makes the physics code
easier to read and write, and reduces the risk of bugs.

\subsubsection{Operators}

\begin{table}[h]
\centering
\caption{Field Assignments} \label{tab:field_assign}
\begin{tabular}[]{ll}
\hline
\hline
Assign from & Function \\
\hline
\code{Field3D = Field3D} & copy all values \\
\code{Field3D = Field2D} & set constant in z \\
\code{Field3D = FieldPerp} & \\
\code{Field3D = bvalue}  & changes a single value \\
\code{Field3D = real} & Sets all point to that value (e.g. set to zero) \\
\hline
\code{Field2D = Field2D} & \\
\code{Field2D = real} & \\
\hline
\code{FieldPerp = FieldPerp} & \\
\hline
\hline
\end{tabular}
\end{table}

\begin{table}[h]
\centering
\caption{Field compound assignment operators} \label{tab:field_cop}
\begin{tabular}[]{ll}
\hline
\hline
Operator & Function \\
\hline
\code{Field3D += Field3D} & Add all points \\
\code{Field3D += Field2D} & Treat 2D field as constant value in z \\
\code{Field2D += Field2D} & \\
\code{FieldPerp += FieldPerp} & \\
\code{FieldPerp += Field3D} & \\
\code{FieldPerp += Field2D} & \\
\code{FieldPerp += real} & \\
\hline
\code{Field3D -= Field3D} & subtract all points \\
\code{Field3D -= Field2D} & subtract constant value in z \\
\code{Field2D -= Field2D} & \\
\code{FieldPerp -= FieldPerp} & \\
\code{FieldPerp -= Field3D} & \\
\code{FieldPerp -= Field2D} & \\
\code{FieldPerp -= real} & \\
\hline
\code{Field3D *= Field3D} & \\
\code{Field3D *= Field2D} & \\
\code{Field3D *= real} & \\
\code{Field2D *= Field2D} & \\
\code{Field2D *= real} & \\
\code{FieldPerp *= FieldPerp} & \\
\code{FieldPerp *= Field3D} & \\
\code{FieldPerp *= Field2D} & \\
\code{FieldPerp *= real} & \\
\hline
\code{Field3D /= Field3D} & \\
\code{Field3D /= Field2D} & \\
\code{Field3D /= real} & \\
\code{Field2D /= Field2D} & \\
\code{Field2D /= real} & \\
\code{FieldPerp /= FieldPerp} & \\
\code{FieldPerp /= Field3D} & \\
\code{FieldPerp /= Field2D} & \\
\code{FieldPerp /= real} & \\
\hline
\code{Field3D \poweq Field3D} & \\
\code{Field3D \poweq Field2D} & \\
\code{Field3D \poweq real} & \\
\code{Field2D \poweq Field2D} & \\
\code{Field2D \poweq real} & \\
\code{FieldPerp \poweq FieldPerp} & \\
\code{FieldPerp \poweq Field3D} & \\
\code{FieldPerp \poweq Field2D} & \\
\code{FieldPerp \poweq real} & \\
\hline
\hline
\end{tabular}
\end{table}

\begin{table}[h]
\centering
\caption{Binary field operations} \label{tab:field_bop}
\begin{tabular}[]{lll}
\hline
\hline
Binary operator & Result & source file \\
\hline
Field2D + Field2D  & Field2D & field2d.c \\
Field2D + Field3D  & Field3D & field2d.c \\
\hline
Field2D - Field2D  & Field2D & field2d.c \\
Field2D - Field3D  & Field3D & field2d.c \\
\hline
Field2D * Field2D  & Field2D & field2d.c \\
Field2D * real     & Field2D & field2d.c \\
real * Field2D     & Field2D & field2d.c (non-member) \\
Field2D * Field3D  & Field3D & field2d.c \\
\hline
Field2D / Field2D  & Field2D & field2d.c \\
Field2D / real     & Field2D & field2d.c \\
real / Field2D     & Field2D & field2d.c (non-member) \\
Field2D / Field3D  & Field3D & field2d.c \\
\hline
Field2D \pow  Field2D & Field2D & field2d.c \\
Field2D \pow real    & Field2D & field2d.c \\
real \pow Field2D    & Field2D & field2d.c (non-member) \\
Field2D \pow Field3D & Field3D & field2d.c \\
\hline
\hline
\end{tabular}
\end{table}

\subsubsection{Memory management}

Many operations require the creation and destruction of intermediate
fields, for example
\begin{verbatim}
Field2D a0
Field3D a, b;
b = (a + a0)^2.0
\end{verbatim}
would require an intermediate Field3D object for \code{a + a0}. It would
be inefficient to keep allocating and freeing memory each time, since
the same operations will be performed for each time-step.

Memory is handled in all Field classes by maintaining a global stack
of memory blocks (since all Fields of a given type are of the same size).
When a new Field object is created, if a memory block is available then it is used,
otherwise a new memory block is allocated. When an object is deleted, it's memory block
is put back onto the stack rather than being freed.

The result of this is that the first time a given code is run, enough memory
is allocated to hold the maximum number of fields used at any given time (since these blocks
will be re-used by several fields). After this first time, no memory needs to be
allocated or freed which should result in a significant increase in speed.

\subsection{Vector2D and Vector3D}

These classes implement 3D (x,y,z) vectors. The \code{Vector2D} class
uses 3 \code{Field2D} objects to store the components and so has no toroidal
(z) variation. The \code{Vector3D} class uses three \code{Field3D} objects.

These vectors can be added/subtracted from each other, and multiplied/divided 
by reals, \code{Field2D} or \code{Field3D} objects. The dot-product is
done by overloading the \code{*} operator, whilst the cross-product is
done using the \code{\pow} operator.

Since all the actual data is stored in \code{Field} objects, all memory management
and operations on the data are done by the \code{Field} methods.

The vector classes are written to be applicable to a general curvilinear coordinate
system; the metric tensor and Jacobian (g** and J in \file{globals.h}) are used
to perform dot and cross-products.

\subsection{Solver}

This is an interface to the time-integration code, currently PVODE.

\begin{itemize}
\item \code{void add({\bf Field2D\&} v, {\bf Field2D\&} F\_v);}
\item \code{void add({\bf Field3D\&} v, {\bf Field3D\&} F\_v);}
\item \code{void setPrecon({\bf rhsfunc} g);}
\item \code{{\bf int} init({\bf real} tstart, {\bf rhsfunc} f, {\bf int} argc, {\bf char**} argv);}
\item \code{{\bf real} run({\bf real} tout);}
\end{itemize}

\subsection{Datafile}

\begin{itemize}
\item \code{void add({\bf int\&} i, {\bf char*} name, {\bf int} grow);}
\item \code{void add({\bf real\&} i, {\bf char*} name, {\bf int} grow);}
\item \code{void add({\bf Field2D\&} f, {\bf char*} name, {\bf int} grow);}
\item \code{void add({\bf Field3D\&} f, {\bf char*} name, {\bf int} grow);}
\item \code{{\bf int} read({\bf char*} filename);}
\item \code{{\bf int} write({\bf char*} filename);}
\item \code{{\bf int} append({\bf char*} filename);}
\end{itemize}

To use this, a set of variables are first added to the object. If the \code{grow}
flag is set, then the number of dimensions in the output is increased by one and extended
each time \code{append} is called. For example, the code:
\begin{verbatim}
Datafile output;
int it;
real time;

output.add(it, ``iteration'', 0);
output.add(time, ``time'', 0);
output.add(time, ``t_array'', 1);

it = 0;
time = 0;

output.write(``file.pdb'');
do {
  it++;
  time = 0.1*( (real) it);
  output.append(``file.pdb'');
}while(it < 9);
\end{verbatim}
would result in an output file containing
\begin{verbatim}
integer  iteration   = 9
real     time        = 0.9   
real     t_array[10] = {0.0, 0.1, ... , 0.9}
\end{verbatim}

\subsection{Communicator}

This is a class which contains all the parallel communication code. The user
first specifies which variables are to be communicated, then whenever necessary
tells the object to perform the communication. This means that several different
sets of variables can be communicated at different times by using several Communicator
objects.

\begin{itemize}
\item \code{void add({\bf Field2D\&} f);} - Specifies that the supplied 2D field is to be communicated.
The memory location of this object should not change over the lifetime of a Communicator object.
\item \code{void add({\bf Field3D\&} f);} - Specifies a 3D field to be communicated.
\item \code{void send();} - Begins the communication process, by posting receives and sending the data
\item \code{void receive();} - Finishes a communication by waiting for all variables to be received.
send() and receive() calls should always be used
in that order in pairs.
\item \code{void run();} - This performs communications by calling send() then receive(). The purpose
of having separate calls is so that a code could perform additional calculations whilst waiting for
data transfer.
\end{itemize}
An example of using the communication object:
\begin{verbatim}
Communicator comms;
Field2D a0;
Fiedl3D a, b;

// Add variables to the communication object
comms.add(a0);
comms.add(a);
comms.add(b);

// Set values
a0 = ...
a = ...
b = ...

comms.run() // perform communications

// more calculations here
\end{verbatim}

\subsection{bstencil and bvalue}

\subsection{OptionFile}

This is an object to read a text file containing settings, formatted like an ini file.
Sections are given in square brackets, and a semicolon comments out the rest of the line.
Whitespace (tabs, spaces, empty lines) is ignored:
\begin{verbatim}
[section name]
name = value  ; comments
\end{verbatim}

Functions provided are:
\begin{itemize}
\item \code{{\bf int} read({\bf char*} filename);}
\item \code{{\bf int} getInt({\bf char*} name, {\bf int\&} val);}
\item \code{{\bf int} getInt({\bf char*} section, {\bf char*} name, {\bf int\&} val);}
\item \code{{\bf int} getReal({\bf char*} name, {\bf real\&} val);}
\item \code{{\bf int} getReal({\bf char*} section, {\bf char*} name, {\bf real\&} val);}
\item \code{{\bf char*} getString({\bf char*} name);}
\item \code{{\bf char*} getString({\bf char*} section, {\bf char*} name);}
\item \code{{\bf int} getBool({\bf char*} name, {\bf int\&} val);}
\item \code{{\bf int} getBool({\bf char*} section, {\bf char*} name, {\bf int\&} val);}
\end{itemize}

To use this class, first read in an option file (\code{read} routine),
then request members by name, passing the value by reference.
These functions return zero if the option exists, except the
\code{getString} routines which return \code{NULL} if the option
isn't found. The \code{getBool} functions treat any string beginning
with a '1', 'y' or 't' as true (1), and any string beginning with
'0', 'n' or 'f' as false (0). 

\subsection{Output}

This class implements a similar variable-argument function to \code{printf}, redirecting
the output to stdout and/or a specified log file. 

\section{Other functions}

\subsection{Grid file reading (\file{grid.cpp})}

The functions to read uedge.grd files are in \file{grid.cpp}, which supplies the following
functions:

\begin{itemize}
\item \code{{\bf int} grid\_read({\bf char} *name);} - Read the geometry values
needed by BOUT++ into global variables.
\item \code{{\bf int} grid\_load2d({\bf Field2D} \&var, {\bf char} *name);} - Read
a single 2D field from the grid file. This is used by the user to read in
initial profiles.

\end{itemize}

\subsection{Differential operators (\file{difops.cpp})}



\subsection{Vector differential operators (\file{vecops.cpp})}

This file implements the following operators

\begin{eqnarray*}
\mathbf{v} &=& \nabla f \\
f &=& \nabla\cdot\mathbf{a} \\
\mathbf{v} &=& \nabla\times\mathbf{a} \\
\mathbf{v} &=& \mathbf{a}\cdot\nabla\mathbf{b}
\end{eqnarray*}



\subsection{Boundary conditions (\file{boundary.cpp})}

These functions currently just set Neumann or Dirichlet boundary conditions,
either by copying values into boundary conditions (zero gradient), or
setting boundary regions to zero (zero value).

\subsubsection{Possible implementation mechanisms}

Regardless of the method used internally, the user interface to
supply boundary conditions should be simple, intuitive, and
require no 

\section{Wish List}

List of things which could (and some which should) be added or improved

\begin{itemize}
\item Optimize Field classes by using copy-on-change. Setting \code{a = b} would just
  cause a's data to point to the same location as b. Only once one of these objects
  was changed would another copy of the data be created. This is also necessary for
  changing the number of grid-points dynamically (see next item)
\item Allow variable number of grid-points on a processor. Putting branch-cuts into stencils,
 it should be possible to run on a single processor. This would allow
  \begin{itemize}
    \item More flexibility in the number of grid-points in the legs
    \item Load balancing by shifting grid-points between neighbouring processors during a run
    \item Adaptive gridding (eventually)
  \end{itemize}
  Since all fields are stored in global arrays (within classes), it should be fairly easy
  to re-size all arrays simultaneously, regardless of whether they are global or user arrays.
\item Create Vector3D and Vector2D classes, and differential operators on them. This would simplify
implementation of ideal MHD or similar equations.
\item Upgrade the Solver class to use a new version of CVODE. Amongst other things, the
new version allows a return code to signal unphysical values.
\item Boundary conditions: Need to think of how to properly implement boundary conditions,
particularly more complicated conditions such as sheath BCs.
\end{itemize}

\section{Notes for users of BOUT}

Significant changes

\begin{itemize}
\item The definition of ZMIN, ZMAX, zlength and dz has been changed. These no longer
need a factor of hthe0. zlength and dz are in radians, and ZMIN/ZMAX are
fractions of $2\pi$, so to simulate $1/5^{th}$ of a torus, $ZMAX-ZMIN = 0.2$.
\end{itemize}

\bibliography{references}
\bibliographystyle{unsrt}

\end{document}
