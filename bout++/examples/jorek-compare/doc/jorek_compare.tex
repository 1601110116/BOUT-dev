%% Manual / derivations for preconditioning in BOUT++

\documentclass[12pt]{article}
\usepackage[nofoot]{geometry}
\usepackage{graphicx}
\usepackage{fancyhdr}
\usepackage{amsfonts}

\usepackage{listings}
\usepackage{color}
\usepackage{textcomp}
\definecolor{listinggray}{gray}{0.9}
\definecolor{lbcolor}{rgb}{0.95,0.95,0.95}
\lstset{
	backgroundcolor=\color{lbcolor},
        language=C++,
	keywordstyle=\bfseries\ttfamily\color[rgb]{0,0,1},
	identifierstyle=\ttfamily,
	commentstyle=\color[rgb]{0.133,0.545,0.133},
	stringstyle=\ttfamily\color[rgb]{0.627,0.126,0.941},
	showstringspaces=false,
	basicstyle=\small,
	numberstyle=\footnotesize,
	numbers=left,
	stepnumber=1,
	numbersep=10pt,
	tabsize=2,
	breaklines=true,
	prebreak = \raisebox{0ex}[0ex][0ex]{\ensuremath{\hookleftarrow}},
	breakatwhitespace=false,
	aboveskip={1.5\baselineskip},
        columns=fixed,
        upquote=true,
        extendedchars=true,
        morekeywords={Field2D,Field3D,Vector2D,Vector3D,real,FieldGroup},
}

%% Modify margins
\addtolength{\oddsidemargin}{-.25in}
\addtolength{\evensidemargin}{-.25in}
\addtolength{\textwidth}{0.5in}
\addtolength{\textheight}{0.25in}
%% SET HEADERS AND FOOTERS

\pagestyle{fancy}
\fancyfoot{}
\renewcommand{\sectionmark}[1]{         % Lower case Section marker style
  \markright{\thesection.\ #1}}
\fancyhead[LE,RO]{\bfseries\thepage}    % Page number (boldface) in left on even
                                        % pages and right on odd pages 
\renewcommand{\headrulewidth}{0.3pt}

\newcommand{\code}[1]{\texttt{#1}}
\newcommand{\file}[1]{\texttt{\bf #1}}

%% commands for boxes with important notes
\newlength{\notewidth}
\addtolength{\notewidth}{\textwidth}
\addtolength{\notewidth}{-3.\parindent}
\newcommand{\note}[1]{
\fbox{
\begin{minipage}{\notewidth}
{\bf NOTE}: #1
\end{minipage}
}}

\newcommand{\pow}{\ensuremath{\wedge} }

\newcommand{\deriv}[2]{\ensuremath{\frac{\partial #1}{\partial #2}}}
\newcommand{\dderiv}[2]{\ensuremath{\frac{\partial^2 #1}{\partial {#2}^2}}}
\newcommand{\Vec}[1]{\ensuremath{\mathbf{#1}}}
\newcommand{\Div}[1]{\ensuremath{\nabla\cdot #1 }}
\newcommand{\Curl}[1]{\ensuremath{\nabla\times #1 }}
\newcommand{\Bvec}{\ensuremath{\underline{B}}}

\begin{document}

\title{JOREK-like reduced MHD equations}

\maketitle

\section{Overview}

Implementation of G.Huysmanns' equations used in JOREK for ELM simulations

\section{Starting equations}

From PPCF {\bf 51} (2009) 124012, the {\bf normalised} equations are:
\begin{eqnarray*}
  \deriv{\rho}{t} &=& -\nabla\left(\rho\underline{v}\right) + \nabla\left(D_\perp\nabla_\perp\rho\right) + S_\rho \\
  \rho\deriv{T}{t} &=& -\rho\underline{v}\cdot\nabla T - \left(\gamma - 1\right)\rho T\nabla\cdot\underline{v} + \nabla\cdot\left(\chi_\perp\nabla_\perp T + \chi_{||}\nabla_{||}T\right) + S_T \\
  \underline{e}_\phi\cdot\nabla\times\left(\rho\deriv{\underline{v}}{t}\right) &=& \underline{e}_\phi\cdot\nabla\times\left(-\rho\left(\underline{v}\cdot\nabla\right)\underline{v} - \nabla\left(\rho T\right) + \underline{J}\times\underline{B} + \mu\nabla^2\underline{v}\right) \\
  \underline{B}\cdot\left(\rho\deriv{\underline{v}}{t}\right) &=& \underline{B}\cdot\left(-\rho\left(\underline{v}\cdot\nabla\right)\underline{v} - \nabla\left(\rho T\right) + \underline{J}\times\underline{B} + \mu\nabla^2\underline{v}\right) \\
  \frac{1}{R^2}\deriv{\psi}{t} &=& \eta\left(T\right)\nabla\cdot\left(\frac{1}{R^2}\nabla_\perp\psi\right) - \underline{B}\cdot\nabla \phi
\end{eqnarray*}
where the velocity $\underline{v} = -R\nabla\phi\times\underline{e}_\phi + v_{||}\underline{B}$

\subsection{Normalisation}

The normalisation used in the above equations uses $\mu_0$ and 
a typical mass density $\rho_0$. Here normalised quantities are given hats
(which were omitted in the above equations). Time is normalised as:
\[
\hat{t} = \frac{t}{\sqrt{\mu_0\rho_0}}
\]
and so velocity is normalised to
\[
\hat{v} = v\sqrt{\mu_0\rho_0}
\]

Pressure and current are normalised to $\mu_0$
\[
\hat{p} = \mu_0 p \qquad \hat{j} = \mu_0 j
\]
The normalised pressure here is given by $\hat{p} = \hat{\rho}\hat{T}$.
Since the pressure in Pascals is given by $p = enT$ where $n$ is the number
density and $T$ is in eV, this gives
\[
\frac{1}{\mu_0}\hat{p} = e\frac{\rho_0\hat{rho}}{m_i} T_0\hat{T}
\]
where $m_i$ is the ion mass. The temperature must therefore be normalised to
\[
T_0 = \frac{m_i}{\mu_0 e \rho_0}
\]
which for $n_0=10^{20}$ gives $T_0\simeq 5\times 10^4$.

The diffusion coefficients are normalised to:
\begin{eqnarray*}
\hat{\eta} &=& \eta\sqrt{\rho_0 / \mu_0} \\
\hat{\mu} &=& \mu\sqrt{\mu_0/\rho_0} \\
\hat{D_\perp} &=& D_\perp\sqrt{\mu_0\rho_0} \\
\hat{\chi_\perp} &=& \chi_\perp\sqrt{\mu_0/\rho_0}
\end{eqnarray*}

\section{Equations solved in BOUT++ model}

Rather than evolving the poloidal flux, the BOUT++ model evolves
the parallel vector potential $\psi = A_{||} / B_0$. The total magnetic field
is therefore given by:
\[
\Bvec = \Bvec_0 + \nabla\times\left(\Bvec_0\psi\right) \simeq \Bvec_0 + \nabla\psi \times\Bvec_0
\]
Either of these expressions can be used, depending on the value of the \texttt{full\_bfield} option. The parallel current is given by
\[
j_{||} = j_{||0} - \frac{B_0}{\mu_0}\nabla_\perp^2\psi
\]
which since $j_{||}$ is normalised to $\mu_0$, and $\psi$ is not normalised gives
\[
\hat{j_{||}} = \hat{j_{||0}} - B_0\nabla_\perp^2\psi
\]

\end{document}

