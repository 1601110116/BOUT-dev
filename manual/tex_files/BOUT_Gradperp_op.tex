\documentclass [11pt]{article}
\usepackage {aip,eqalign}
\pagestyle{myheadings}
\topmargin -0.50truein
\oddsidemargin 0.truein
\textheight 9.00truein
\textwidth 6.5truein
\renewcommand{\baselinestretch}{1.5}
\parskip 0.10in

\font\tenbit=cmmib10 \font\sevenbit=cmmib7 \font\fivebit=cmmib5
     \newfam\bitfamno
     \textfont\bitfamno=\tenbit   \scriptfont\bitfamno=\sevenbit
                                  \scriptscriptfont\bitfamno=\fivebit
\def\bfit{\fam\bitfamno \tenbit}  %bold face italic font, for text and mthmode.

%\mathchardef\kappa="7114
\mathchardef\bfkappa="0114

\usepackage[]{graphicx}

% Add an index
\usepackage{makeidx}
\makeindex

\def\u{{\bf u}}
\def\B{{\bf B}}
\def\E{{\bf E}}
\def\J{{\bf J}}
\def\v{{\bf v}}
\def\F{{\bf F}}
\def\R{{\bf R}}
\def\bx{B_x}
\def\curl{\nabla \times}
\def\div{\nabla \cdot}
\def\grad{\nabla}
\def\lap{\nabla^2}
\def\pt{\partial_t}
\def\pr{\partial_r}
\def\pz{\partial_z}
\def\bl{\item{$\bullet$}}
\def\sp{\item{-}}
\def\dz{{d \over dz}}
\def\rp{R^\prime}
\def\fc{{\cal F}}
\def\rc{{\cal R}}
\def\dec{\Delta {\cal E}}
\def\pd{\partial}
\def\deriv#1#2{{d#1\over d#2}}
\def\pderiv#1#2{{\partial#1\over \partial#2}}
\def\parder#1#2{{\partial#1\over\partial#2}}
\def\pardersq#1#2{{\partial^2#1\over\partial#2^2}}
\def\av#1{\left\langle#1\right\rangle}
\def\cosa{\cos \alpha}
\def\sina{\sin \alpha}
\def\epei{\epsilon_{e,i}}
\def\epnuie{\epsilon_{\nu i, \nu e}}
\def\vp{v_\perp}
\def\vh{\hat v}
\def\vhp{{\hat v}_\perp}
\def\epr{\epsilon_{\rho}}
\def\epd{\epsilon_{\Delta}}
\def\epl{\epsilon_{\lambda}}
\def\epnu{\epsilon_{\nu}}
\def\exb{{\bf E \! \times \! B}}
\def\iv{{\bf \hat i}}

\def\ptensor{\font\sf=cmssbx10 \sl P}

\begin{document}

\title{\bf Geometry and Differential Operator }
\author{X.~Q.~Xu}
%\author{X.~Q.~Xu and A.~Dimits}
 \date{\today}
\maketitle

\begin{abstract}
The memo is written for understanding approximations 
when the field-aligned coordinates are used for the 
conventional turbulence ordering ($k_\|\ll k_\perp$).
Special cares must be taken in order to keep
$m\ne0$ for $n=0$ modes. Otherwise, in the $(m,n)$ phase space, 
the $n=0$ and $m>0$ modes should be suppressed to 
keep turbulence ordering ($k_\|\ll k_\perp$) consistent for $n=0$ modes.
\end{abstract}

\section{Geometry}

In a axisymmetric toroidal system, the magnetic field can be expressed as
$${\bf B}=I(\psi)\nabla\zeta+\nabla\zeta\times\nabla\psi,$$
where $\psi$ is the poloidal flux, 
$\theta$ is the poloidal angle-like coordinate, and $\zeta$ is the toroidal angle. Here, $I(\psi)=RB_t$.
The two important geometrical parameters are: the curvature, $\bf \kappa$, and 
the local pitch, $\nu(\psi,\theta)$,
$$\nu(\psi,\theta)= {I(\psi){\bf \cal J}/R^2}.$$
The local pitch $\nu(\psi,\theta)$ is related to the MHD safety q 
by $\hat q(\psi)={2\pi}^{-1}\oint\nu(\psi,\theta) d\theta$ in the closed flux surface region, and  
$\hat q(\psi)={2\pi}^{-1}\int_{inboard}^{outboard}\nu(\psi,\theta) d\theta$ in the scrape-off-layer. 
Here ${\bf \cal J}=(\nabla\psi\times\nabla\theta\cdot\nabla\zeta)^{-1}$
is the coordinate Jacobian, 
$R$ is the major radius, and $Z$ is the vertical position.

\section{Geometry and Differential Operators}

In a axisymmetric toroidal system, the magnetic field can be expressed as
${\bf B}=I(\psi)\nabla\zeta+\nabla\zeta\times\nabla\psi$, where $\psi$ is the poloidal flux, 
$\theta$ is the poloidal angle-like coordinate, and $\zeta$ is the toroidal angle. Here, $I(\psi)=RB_t$.
The two important geometrical parameters 
are: the curvature, $\bf \kappa$, and 
the local pitch, $\nu(\psi,\theta)$, and $\nu(\psi,\theta)= {I(\psi){\bf \cal J}/R^2}$. 
The local pitch $\nu(\psi,\theta)$ is related to the MHD safety q 
by $\hat q(\psi)={2\pi}^{-1}\oint\nu(\psi,\theta) d\theta$ in the closed flux surface region, and  
$\hat q(\psi)={2\pi}^{-1}\int_{inboard}^{outboard}\nu(\psi,\theta) d\theta$ in the scrape-off-layer. 
Here ${\bf \cal J}=(\nabla\psi\times\nabla\theta\cdot\nabla\zeta)^{-1}$
is the coordinate Jacobian, 
$R$ is the major radius, and $Z$ is the vertical position.

\newpage
\subsection{Differential Operators}
For such an axisymmetric equilibrium the metric coefficients are only functions of $\psi$ and $\theta$. Three spatial 
differential operators appear in the equations given as: ${\bf v_E}\cdot\nabla_\perp$, $\nabla_\|$ and 
$\nabla_\perp^2$.
\begin{eqnarray}
\nabla_\|&=&{\bf b_0}\cdot\nabla={1\over {\cal J}B}{\partial\over\partial\theta}+{I\over BR^2}{\partial\over\partial\zeta}={B_p\over hB}{\partial\over\partial\theta}+{B_t\over RB}{\partial\over\partial\zeta}, \\
{\cal J}\nabla^2&=&
{\partial\over\partial\psi}\left({\cal J}J_{11}{\partial\over\partial\psi}\right)
+{\partial\over\partial\psi}\left({\cal J}J_{12}{\partial\over\partial\theta}\right) \nonumber\\
&+&{\partial\over\partial\theta}\left({\cal J}J_{22}{\partial\over\partial\theta}\right)
+{\partial\over\partial\theta}\left({\cal J}J_{12}{\partial\over\partial\psi}\right)  \nonumber\\
&+&{1\over R^2}{\partial^2\over\partial\zeta^2}. \\
\nabla_\|^2&=&{\bf b}_0\cdot\nabla({\bf b}_0\cdot\nabla)={1\over {\cal J}B}{\partial\over\partial\theta}\left({1\over {\cal J}B}{\partial\over\partial\theta}\right)
+{1\over {\cal J}B}{\partial\over\partial\theta}\left({B_t\over RB}{\partial\over\partial\zeta}\right) \\
&+&{B_t\over {\cal J}RB^2}{\partial^2\over\partial\theta\partial\zeta}
+\left({B_t\over {\cal J}RB}\right)^2{\partial^2\over\partial\zeta^2}, \\
\nabla_\perp^2\Phi&=&-\nabla\cdot[{\bf b}\times({\bf b}\times\nabla\Phi)]=\nabla^2\Phi-(\nabla\cdot{\bf b})({\bf b}\cdot\nabla\Phi)-\nabla_\|^2\Phi
\end{eqnarray}
where the coordinate Jacobian and metric coefficients are defined as following:
\begin{eqnarray}
{\cal J}&=&\nabla\psi\times\nabla\theta\cdot\nabla\zeta={h\over B_p}, \\
h&=&\sqrt{Z_\theta^2+R_\theta^2}, \\
J_{11}&=&|\nabla\psi|^2={R^2\over {\cal J}^2}(Z_\theta^2+R_\theta^2), \\
J_{12}&=&J_{21}=\nabla\psi\cdot\nabla\theta=-{R^2\over {\cal J}^2}(Z_\theta Z_\psi+R_\psi R_\theta), \\
J_{13}&=&J_{31}=0, \\
J_{22}&=&|\nabla\theta|^2={R^2\over {\cal J}^2}(Z_\psi^2+R_\psi^2), \\
J_{23}&=&J_{32}=0, \\
J_{33}&=&|\nabla\zeta|^2={1\over R^2}.
\end{eqnarray}

\subsection{Concentric circular cross section inside the separatrix without the SOL}
For concentric circular cross section inside the separatrix without the SOL, the differential operators are reduced to:
\begin{eqnarray}
R&=&R_0+rcos\theta, \\
Z&=&rsin\theta, \\
B_t&=&{B_{t0}R_0\over R}, \\
B_p&=&{1\over R}{\partial\psi\over\partial r}, \\
R_\psi&=&{cos\theta\over RB_p}, \\
R_\theta&=&-rsin\theta, \\
Z_\psi&=&{sin\theta\over RB_p}, \\
Z_\theta&=&rcos\theta, \\
{\cal J}&=&{r\over B_p}, \\
h&=&r, \\
J_{11}&=&|\nabla\psi|^2=r^2B_p^2, \\
J_{12}&=&J_{21}=\nabla\psi\cdot\nabla\theta=0,\\
J_{13}&=&J_{31}=0, \\
J_{22}&=&|\nabla\theta|^2={1\over r^2}, \\
J_{23}&=&J_{32}=0, \\
J_{33}&=&|\nabla\zeta|^2={1\over R^2},\\
\nabla^2&\simeq&{1\over r}{\partial\over\partial r}\left(r{\partial\over\partial r}\right)+{1\over r^2}{\partial^2\over\partial \theta^2}+{1\over R^2}{\partial^2\over\partial \zeta^2}
\end{eqnarray}

\newpage
\subsection{\bf  Field-aligned coordinates with $\theta$ as the coordinate along the field line}
A suitable coordinate mapping between  field-aligned ballooning coordinates ($x$, $y$, $z$) 
and the usual flux coordinates ($\psi$, $\theta$, $\zeta$)  is
\begin{eqnarray}
x&=&\psi-\psi_s, \nonumber \\
y&=&\theta, \nonumber \\
z&=&\zeta-\int_{\theta_0}^\theta \nu(x,y)dy.
\end{eqnarray}
as shown in Fig.~1.
The covering area given by the square ABCD in the usual flux coordinates is the same as the parallelogram ABEF in the 
field-aligned coordinates. 
The magnetic separatrix is denoted by $\psi=\psi_s$. 
In this choice of coordinates, $x$ is a flux surface label, $y$, the poloidal angle, is also
the coordinate along the field line, and $z$ is a field line label within the flux surface.


The coordinate Jacobian and metric coefficients are defined as following:
\begin{eqnarray}
{\cal J}&=&\nabla\psi\times\nabla\theta\cdot\nabla\zeta={h\over B_p}, \\
h&=&\sqrt{Z_\theta^2+R_\theta^2}, \\
{\cal J}_{11}&=&|\nabla x|^2={R^2\over {\cal J}^2}(Z_\theta^2+R_\theta^2), \\
{\cal J}_{12}&=&{\cal J}_{21}=\nabla x\cdot\nabla y=-{R^2\over {\cal J}^2}(Z_\theta Z_\psi+R_\psi R_\theta), \\
{\cal J}_{22}&=&|\nabla y|^2={R^2\over {\cal J}^2}(Z_\psi^2+R_\psi^2), \\
{\cal J}_{13}&=&{\cal J}_{31}=\nabla x\cdot\nabla z=-\nu\nabla x\cdot\nabla y-|\nabla x|^2\left(\int_{y_0}^y {\partial \nu(x,y)\over\partial\psi}dy\right)=-|\nabla x|^2I_s, \\
{\cal J}_{23}&=&{\cal J}_{32}=\nabla y\cdot\nabla z=-\nu|\nabla y|^2-\nu\nabla x\cdot\nabla y\left(\int_{y_0}^y {\partial \nu(x,y)\over\partial\psi}dy\right), \\
{\cal J}_{33}&=&|\nabla z|^2=\left |\nabla\zeta-\nu\nabla \theta-\nabla\psi\left(\int_{y_0}^y {\partial \nu(x,y)\over\partial\psi}dy\right)\right |^2, \\
I_s &=&  {{\cal J}_{12}\over|\nabla\psi|^2}\nu(x,y)+\left(\int_{y_0}^y {\partial \nu(x,y)\over\partial\psi}dy\right).
\end{eqnarray}
Here $h$ is the local minor radius, $I_s$ is the integrated local shear, and $y_0$ is an arbitrary integration parameter, which, depending on the choice of Jacobian, determines the location where $ I_s=0$. 
The disadvantage of this choice of coordinates is that
the Jacobian diverges near the X-point as $B_p\rightarrow 0$
and its effect spreads over the entire flux surafces
near the separatrix as the results of coordinate transform $z$.
Therefore a better set of coordinates is needed for X-point divertor geometry.
The derivatives are obtained from the chain rule as follows:
\begin{eqnarray}
{d\over d\psi}&=&{\partial\over \partial x} - \left(\int_{y_0}^y {\partial \nu(x,y)\over\partial\psi}dy\right){\partial\over \partial z},   \\ 
{d\over d\theta}&=&{\partial\over \partial y} - \nu(x,y){\partial\over \partial z},   \\ 
{d\over d\zeta}&=&{\partial\over \partial z}.
\end{eqnarray}

In the field-aligned ballooning coordinates, the parallel differential operator is simple, involving only one coordinate $y$
\begin{eqnarray}
\partial_\|^0 &=&  {\bf b}_0\cdot\nabla_\|=\left({B_p\over hB}\right){\partial\over\partial y}.
\end{eqnarray}
which requires a few grid points.
The total axisymmetric drift operator becomes

The perturbed ${\bf E}\times {\bf B}$ drift operator becomes
\begin{eqnarray}
{\delta\bf v_E}\cdot\nabla_\perp&=&
{c\over BB_\|^*}\left\{
-{I\over J}{\partial\langle\delta\phi\rangle\over\partial\theta}
+{B_p^2}
{\partial\langle\delta\phi\rangle\over\partial z}
\right\}{\partial\over\partial\psi} \nonumber\\
&+&{c\over BB_\|^*}\left\{{I\over{\cal J}}
{\partial\langle\delta\phi\rangle\over\partial\psi}
+{{\cal J}_{12}\over R^2}
{\partial\langle\delta\phi\rangle\over\partial z}
\right\}{\partial\over\partial\theta} \nonumber\\
&-&{c\over BB_\|^*}\left\{B_p^2
{\partial\langle\delta\phi\rangle\over\partial\psi}
+{{\cal J}_{12}\over R^2}
{\partial\langle\delta\phi\rangle\over\partial\theta}
\right\}{\partial\over\partial z},
\end{eqnarray}
when the conventional turbulence ordering ($k_\|\ll k_\perp$) is used, the 
perturbed ${\bf E}\times {\bf B}$ drift operator can be further reduced to a simple form
\begin{eqnarray}
{\delta\bf v_E}\cdot\nabla_\perp&=&
{cB\over B_\|^*}\left(
{\partial\langle\delta\phi\rangle\over\partial z}{\partial\over\partial x}
-{\partial\langle\delta\phi\rangle\over\partial x}{\partial\over\partial z}\right)
\end{eqnarray}
where $\partial/\partial\theta\simeq -\nu\partial/\partial z$ is used.
In the perturbed ${\bf E}\times {\bf B}$ drift operator the poloidal and radial derivatives 
are written in 
the usual flux $(\psi,\theta,\zeta)$ coordinates in order to have various options for valid discretizations. 
The general Laplacian operator for potential is
\begin{eqnarray}
{\cal J}\nabla^2\Phi&=&{\partial\over\partial x}\left({\cal J}{\cal J}_{11}{\partial\Phi\over\partial x}
+{\cal J}{\cal J}_{12}{\partial\Phi\over\partial y}
+{\cal J}{\cal J}_{13}{\partial\Phi\over\partial z}\right) \nonumber\\
&+&{\partial\over\partial y}\left({\cal J}{\cal J}_{21}{\partial\Phi\over\partial x}
+{\cal J}{\cal J}_{22}{\partial\Phi\over\partial y}
+{\cal J}{\cal J}_{23}{\partial\Phi\over\partial z}\right) \nonumber\\
&+&{\partial\over\partial z}\left({\cal J}{\cal J}_{31}{\partial\Phi\over\partial x}
+{\cal J}{\cal J}_{32}{\partial\Phi\over\partial y}
+{\cal J}{\cal J}_{33}{\partial\Phi\over\partial z}\right).
\end{eqnarray}
The general perpendicular Laplacian operator for potential is
\begin{eqnarray}
{\cal J}\nabla_\perp^2\Phi&=&{\partial\over\partial x}\left({\cal J}{\cal J}_{11}{\partial\Phi\over\partial x}
+{\cal J}{\cal J}_{12}{\partial\Phi\over\partial y}
+{\cal J}{\cal J}_{13}{\partial\Phi\over\partial z}\right) \nonumber\\
&+&{\partial\over\partial y}\left({\cal J}{\cal J}_{21}{\partial\Phi\over\partial x}
+{\cal J}{\cal J}_{22}{\partial\Phi\over\partial y}
+{\cal J}{\cal J}_{23}{\partial\Phi\over\partial z}\right) \nonumber\\
&+&{\partial\over\partial z}\left({\cal J}{\cal J}_{31}{\partial\Phi\over\partial x}
+{\cal J}{\cal J}_{32}{\partial\Phi\over\partial y}
+{\cal J}{\cal J}_{33}{\partial\Phi\over\partial z}\right) \nonumber\\
&-&\left({B_p\over hB}\right){\partial\over\partial y}
\left[\left({B_p\over hB}\right){\partial\Phi\over\partial y}\right] \nonumber\\
&-&\left({B_p\over hB}\right)^2{\partial\ln B\over\partial y}{\partial\Phi\over\partial y}.
\end{eqnarray}
The general perpendicular Laplacian operator for axisymmetric potential $\Phi_0(x,y)$ is
\begin{eqnarray}
{\cal J}\nabla_\perp^2\Phi_0&=&{\partial\over\partial x}\left({\cal J}{\cal J}_{11}{\partial\Phi_0\over\partial x}
+{\cal J}{\cal J}_{12}{\partial\Phi_0\over\partial y}\right) \nonumber\\
&+&{\partial\over\partial y}\left({\cal J}{\cal J}_{21}{\partial\Phi_0\over\partial x}
+{\cal J}{\cal J}_{22}{\partial\Phi_0\over\partial y}\right) \nonumber\\
&-&\left({B_p\over hB}\right){\partial\over\partial y}
\left[\left({B_p\over hB}\right){\partial\Phi_0\over\partial y}\right]  \nonumber\\
&-&\left({B_p\over hB}\right)^2{\partial\ln B\over\partial y}{\partial\Phi\over\partial y}.
\end{eqnarray}
For the perturbed potential $\delta\phi$, we can drop the $\partial/\partial y$ terms in Eq.~(69) 
due to the elongated nature of the turbulence ($k_\|/k_\perp\ll1$).
The general perpendicular Laplacian operator for perturbed potential $\delta\phi$ reduces to
\begin{eqnarray}
{\cal J}\nabla_\perp^2\delta\phi&=&
{\partial\over\partial x}\left({\cal J}{\cal J}_{11}{\partial\delta\phi\over\partial x}
+{\cal J}{\cal J}_{13}{\partial\delta\phi\over\partial z}\right) \nonumber\\
&+&{\partial\over\partial z}\left({\cal J}{\cal J}_{31}{\partial\delta\phi\over\partial x}
+{\cal J}{\cal J}_{33}{\partial\delta\phi\over\partial z}\right).
\end{eqnarray}
If the non-split potential $\Phi$ is a preferred option, the gyrokinetic Poisson equation (18) and the 
general perpendicular Laplacian operator Eq.~(69) have to be used. Then the assumption 
$k_\|/k_\perp\ll1$ is not used to simplify the perpendicular Laplacian operator.

\end{document}
