%% Manual / derivations for preconditioning in BOUT++

\documentclass[12pt]{article}
\usepackage[nofoot]{geometry}
\usepackage{graphicx}
\usepackage{fancyhdr}
\usepackage{amsfonts}

\usepackage{listings}
\usepackage{color}
\usepackage{textcomp}

% Add an index
\usepackage{makeidx}
\makeindex

\definecolor{listinggray}{gray}{0.9}
\definecolor{lbcolor}{rgb}{0.95,0.95,0.95}
\lstset{
	backgroundcolor=\color{lbcolor},
        language=C++,
	keywordstyle=\bfseries\ttfamily\color[rgb]{0,0,1},
	identifierstyle=\ttfamily,
	commentstyle=\color[rgb]{0.133,0.545,0.133},
	stringstyle=\ttfamily\color[rgb]{0.627,0.126,0.941},
	showstringspaces=false,
	basicstyle=\small,
	numberstyle=\footnotesize,
	numbers=left,
	stepnumber=1,
	numbersep=10pt,
	tabsize=2,
	breaklines=true,
	prebreak = \raisebox{0ex}[0ex][0ex]{\ensuremath{\hookleftarrow}},
	breakatwhitespace=false,
	aboveskip={1.5\baselineskip},
        columns=fixed,
        upquote=true,
        extendedchars=true,
        morekeywords={Field2D,Field3D,Vector2D,Vector3D,real,FieldGroup},
}

%% Modify margins
\addtolength{\oddsidemargin}{-.25in}
\addtolength{\evensidemargin}{-.25in}
\addtolength{\textwidth}{0.5in}
\addtolength{\textheight}{0.25in}
%% SET HEADERS AND FOOTERS

\pagestyle{fancy}
\fancyfoot{}
\renewcommand{\sectionmark}[1]{         % Lower case Section marker style
  \markright{\thesection.\ #1}}
\fancyhead[LE,RO]{\bfseries\thepage}    % Page number (boldface) in left on even
                                        % pages and right on odd pages 
\renewcommand{\headrulewidth}{0.3pt}

\newcommand{\code}[1]{\texttt{#1}}
\newcommand{\file}[1]{\texttt{\bf #1}}

%% commands for boxes with important notes
\newlength{\notewidth}
\addtolength{\notewidth}{\textwidth}
\addtolength{\notewidth}{-3.\parindent}
\newcommand{\note}[1]{
\fbox{
\begin{minipage}{\notewidth}
{\bf NOTE}: #1
\end{minipage}
}}

\newcommand{\pow}{\ensuremath{\wedge} }
\newcommand{\poweq}{\ensuremath{\wedge =} }

\newcommand{\deriv}[2]{\ensuremath{\frac{\partial #1}{\partial #2}}}
\newcommand{\dderiv}[2]{\ensuremath{\frac{\partial^2 #1}{\partial {#2}^2}}}
\newcommand{\Vpar}{\ensuremath{V_{||}}}
\newcommand{\Gradpar}{\ensuremath{\partial_{||}}}
\newcommand{\Divpar}{\ensuremath{\nabla_{||}}}
\newcommand{\DivXgradX}[2]{\ensuremath{\nabla_\psi\left(#1\partial_\psi #2\right)}}
\newcommand{\DivParGradPar}[2]{\ensuremath{\nabla_{||}\left(#1\partial_{||} #2\right)}}

\newcommand{\apar}{\ensuremath{A_{||}}}
\newcommand{\hthe}{\ensuremath{h_\theta}}
\newcommand{\Bp}{\ensuremath{B_\theta}}
\newcommand{\Bt}{\ensuremath{B_\zeta}}

\newcommand{\Vec}[1]{\ensuremath{\mathbf{#1}}}
\newcommand{\bvec}{\Vec{b}}
\newcommand{\kvec}{\Vec{\kappa}}
\newcommand{\vvec}{\Vec{v}}
\newcommand{\bxk}{\bvec_0\times\kvec_0\cdot\nabla}
\newcommand{\Bvec}{\Vec{B}}
\newcommand{\Bbar}{\overline{B}}
\newcommand{\Lbar}{\overline{L}}
\newcommand{\Tbar}{\overline{T}}
\newcommand{\Jvec}{\Vec{J}}
\newcommand{\Jpar}{J_{||}}
\newcommand{\delp}{\nabla_\perp^2}
\newcommand{\Div}[1]{\ensuremath{\nabla\cdot #1 }}
\newcommand{\Curl}[1]{\ensuremath{\nabla\times #1 }}
\newcommand{\rbp}{\ensuremath{R\Bp}}
\newcommand{\rbpsq}{\ensuremath{\left(\rbp\right)^2}}
\newcommand{\Jac}{\ensuremath{\mathbb{J}}}
\newcommand{\vE}{\ensuremath{\Vec{v}_E}}

\begin{document}

\title{BOUT++ preconditioning}
\author{B.Dudson, University of York}

\maketitle

\tableofcontents

\section{Introduction}

This manual describes some of the ways BOUT++ could (and in some cases does)
support preconditioning, Jacobian calculations and other methods to speed up
simulations. This manual assumes that you're familiar with how BOUT++ works
internally.

{\bf Some notation}: The ODE being solved is of the form
\[
\deriv{\Vec{f}}{t} = \Vec{F}\left(\Vec{f}\right)
\]
Here the state vector $f = \left(f_0, f_1, f_2, \ldots\right)^T$ is a vector
containing the evolving (3D) variables $f_i\left(x,y,z\right)$.

The Jacobian of this system is then
\[
\Jac = \deriv{\Vec{F}}{\Vec{f}}
\]

The order of the elements in the vector $\Vec{f}$ is determined in the
solver code and SUNDIALS, so here just assume that there exists a
map $\mathbb{I}$ between a global index $k$ and (variable, position) i.e. $\left(i,x,y,z\right)$
\[
\mathbf{I} : \left(i,x,y,z\right) \mapsto k
\]
and it's inverse
\[
\mathbf{I}^{-1} : k \mapsto \left(i,x,y,z\right)
\]

Some problem-specific operations which can be used to speed up the timestepping
\begin{enumerate}
\item Jacobian-vector multiply: Given a vector, multiply it by $\Jac$
\item Preconditioner multiply: Given a vector, multiply by an approximate
  inverse of $\mathbb{M} = \mathbb{I} - \gamma\mathbb{J}$
\item Calculate the stencils i.e. non-zero elements in $\Jac$
\item Calculate the non-zero elements of $\Jac$
\end{enumerate}

\section{Physics problems}

Some interesting physics problems of increasing difficulty

\subsection{Resistive drift-interchange instability}

A ``simple'' test problem of 2 fields, which results in non-trivial
turbulent results. Supports resistive drift wave and interchange
instabilities.
\begin{eqnarray*}
\deriv{N_i}{t} + \vE\cdot\nabla N_i &=& 0 \\
\deriv{\omega}{t} + \vE\cdot\nabla\omega &=& 2\omega_{ci}\bvec\times\kappa\cdot\nabla P + N_iZ_i e\frac{4\pi V_A^2}{c^2}\nabla_{||}j_{||} \\
\nabla_\perp^2\omega / N_i &=& \phi \\
0.51\nu_{ei}j_{||} &=& \frac{e}{m_e}\partial_{||}\phi + \frac{T_e}{N_i m_e}\partial_{||} N_i
\end{eqnarray*}

\subsection{Reduced 3-field MHD}

This is a 3-field system of pressure $P$, magnetic
flux $\psi$ and vorticity $U$:
\[
\Vec{f} = \left(\begin{array}{c}
P \\
\psi \\
U
\end{array}\right)
\]

\begin{eqnarray*}
  \deriv{\psi}{t} &=& -\frac{1}{B_0}\nabla_{||}\phi \\
  &=& -\frac{1}{B_0}\left[\bvec_0 - \left(\bvec_0\times\nabla\psi\right)\right]\cdot\nabla\phi \\
  &=& -\frac{1}{B_0}\bvec_0\cdot\nabla\phi - \frac{1}{B_0}\left(\bvec_0\times\nabla\phi\right)\cdot\nabla\psi \\
\Rightarrow \frac{d \psi}{dt} &=& -\frac{1}{B_0}\bvec_0\cdot\nabla \phi
\end{eqnarray*}

The coupled set of equations to be solved are therefore
\begin{eqnarray}
\frac{1}{B_0}\delp\phi &=& U \\
\left(\deriv{}{t} + \vvec_E\cdot\nabla\right)\psi &=& -\frac{1}{B_0}\bvec_0\cdot\nabla\phi \\
\left(\deriv{}{t} + \vvec_E\cdot\nabla\right)P &=& 0 \\
\left(\deriv{}{t} + \vvec_E\cdot\nabla\right)U &=& \frac{1}{\rho}B_0^2\left[\bvec_0 - \left(\bvec_0\times\nabla\psi\right)\right]\cdot\left(\frac{J_{||0}}{B_0} - \frac{1}{\mu_0}\delp\psi\right) \nonumber \\
&+& \frac{1}{\rho}\bxk P \\
\vvec_E &=& \frac{1}{B_0}\bvec_0\times\nabla\phi
\end{eqnarray}

The Jacobian of this system is therefore:
\begin{equation}
\mathbb{J} = 
\left[ \begin{array}{c|c|c}
\color{blue}{-\vvec_E\cdot\nabla} & 0 & \left[\bvec_0\times\nabla\left(P_0 + \color{blue}{P}\right)\cdot\nabla\right]\nabla_\perp^{-2} \\
\hline
0 & \color{blue}{-\vvec_E\cdot\nabla} & \left(\bvec_0\cdot\nabla\right)\nabla_\perp^{-2}  \\
\hline
2\bxk  & -\frac{B_0^2}{\mu_0\rho}\left(\bvec_0 \color{blue}{-\bvec_0\times\nabla\psi}\right)\cdot\nabla\delp  & \color{blue}{-\vvec_E\cdot\nabla} \\
 & + \frac{B_0^2}{\rho}\left[\bvec_0\times\nabla\left(\frac{J_{||0}}{B_0}\right)\right]\cdot\nabla & \\
 & + \color{blue}{\frac{B_0^2}{\mu_0\rho}\nabla\left(\delp\psi\right)\cdot\left(\bvec_0\times\nabla\right)} & 
\end{array}\right]
\label{eq:mhdjacobian}
\end{equation}
Where the blue terms are only included in nonlinear simulations.

This Jacobian has large dense blocks because of the Laplacian inversion terms
(involving $\nabla_\perp^{-2}$ which couples together all points in an
X-Z plane. The way to make $\Jac$ sparse is to solve $\phi$ as a constraint
(using e.g. the IDA solver) which moves the Laplacian inversion to the
preconditioner.

\subsection{Solving $\phi$ as a constraint}

The evolving state vector becomes
\[
\Vec{f} = \left(\begin{array}{c}
P \\
\psi \\
U \\
\phi
\end{array}\right)
\]

\subsection{UEDGE equations}

The UEDGE benchmark is a 4-field model with the following equations:

\begin{eqnarray*}
\deriv{N_i}{t} + \Vpar\partial_{||}N_i &=& -N_i\nabla_{||}\Vpar +\nabla_\psi\left(D_\perp \partial_\psi N_i\right) \\
\deriv{\left(N_i\Vpar\right)}{t} + \Vpar\partial_{||}\left(N_i\Vpar\right) &=& -\partial_{||}P + \nabla_\psi\left(N_i\mu_\perp\partial_\psi\Vpar\right) \\
\frac{3}{2}\deriv{}{t}\left(N_iT_e\right) &=& \nabla_{||}\left(\kappa_e\partial_{||}T_e\right) + \nabla_\psi\left(N_i\chi_\perp\partial_\perp T_e\right) \\
\frac{3}{2}\deriv{}{t}\left(N_iT_i\right) &=& \nabla_{||}\left(\kappa_i\partial_{||}T_i\right) + \nabla_\psi\left(N_i\chi_\perp\partial_\perp T_i\right)
\end{eqnarray*}

This set of equations is good in that there is no inversion needed,
and so the Jacobian is sparse everywhere. The state vector is
\[
\Vec{f} = \left(\begin{array}{c}
N_i \\
\Vpar \\
T_e \\
T_i \\
\end{array}\right)
\]

The Jacobian is:
\begin{equation}
\mathbb{J} = 
\left( \begin{array}{c|c|c|c}
  -\Vpar\partial_{||} - \nabla_{||}\Vpar + \nabla_\psi D_\perp\partial_\psi & -\partial_{||}N_i - N_i\nabla_{||} & 0 & 0 \\
-\frac{1}{N_i}\deriv{\Vpar}{t} - \frac{1}{N_i}\Vpar\Jac_{N_iN_i} & & &
\end{array}\right)
\end{equation}


If instead the state vector is
\[
\Vec{f} = \left(\begin{array}{c}
N_i \\
N_i\Vpar \\
N_iT_e \\
N_iT_i \\
\end{array}\right)
\]
then the Jacobian is



\subsection{2-fluid turbulence}


\section{Jacobian-vector multiply}

This is currently implemented into the CVODE (SUNDIALS) solver.

\section{Preconditioner-vector multiply}



\subsection{Reduced 3-field MHD}
The matrix $\mathbb{M}$ to be inverted can therefore be written
\begin{equation}
\mathbb{M} = 
\left[ \begin{array}{ccc}
\mathbb{D} & 0 & \mathbb{U}_P \\
0 & \mathbb{D} & \mathbb{U}_\psi \\
\mathbb{L}_P & \mathbb{L}_\psi & \mathbb{D}
\end{array}\right]
\end{equation}
where
\[
\mathbb{D} = \mathbb{I} \color{blue}{+ \gamma\vvec_E\cdot\nabla}
\]
For small flow velocities, the inverse of $\mathbb{D}$ can be approximated using the Binomial theorem:
\begin{equation}
\mathbb{D}^{-1} \simeq \mathbb{I} \color{blue}{- \gamma\vvec_E\cdot\nabla}
\label{eq:dapprox}
\end{equation}
Following \cite{chacon-2008, chacon-2002}, $\mathbb{M}$ can be re-written as
\[
\mathbb{M} = 
\left[ \begin{array}{cc}
\mathbb{E} & \mathbb{U} \\
\mathbb{L} & \mathbb{D}
\end{array}\right] \qquad \mathbb{E} = 
\left[ \begin{array}{cc}
\mathbb{D} & 0 \\
0 & \mathbb{D}
\end{array}\right] \qquad \mathbb{U} =
\left(\begin{array}{c}
\mathbb{U}_P \\
\mathbb{U}_\psi
\end{array}\right) \qquad \mathbb{L} = \left(\mathbb{L}_P \quad \mathbb{L}_\psi\right)
\]
The Schur factorization of $\mathbb{M}$ yields \cite{chacon-2008}
\[
\mathbb{M}^{-1} = 
\left[ \begin{array}{cc}
\mathbb{E} & \mathbb{U} \\
\mathbb{L} & \mathbb{D}
\end{array}\right]^{-1} = 
\left[ \begin{array}{cc}
\mathbb{I} & -\mathbb{E}^{-1}\mathbb{U} \\
0 & \mathbb{I}
\end{array}\right]
\left[ \begin{array}{cc}
\mathbb{E}^{-1} & 0 \\
0 & \mathbb{P}_{Schur}^{-1}
\end{array}\right]
\left[ \begin{array}{cc}
\mathbb{I} & 0 \\
-\mathbb{L}\mathbb{E}^{-1} & \mathbb{I}
\end{array}\right]
\]
Where $\mathbb{P}_{Schur} = \mathbb{D} - \mathbb{L}\mathbb{E}^{-1}\mathbb{U}$ is the Schur complement. Note that
this inversion is exact so far.
Since $\mathbb{E}$ is block-diagonal, and $\mathbb{D}$ can be easily approximated using equation \ref{eq:dapprox},
this simplifies the problem to inverting $\mathbb{P}_{Schur}$, which is much smaller than $\mathbb{M}$.

A possible approximation to $\mathbb{P}_{Schur}$ is to neglect:
\begin{itemize}
  \item All drive terms
    \begin{itemize}
    \item the curvature term $\mathbb{L}_P$
    \item the $J_{||0}$ term in $\mathbb{L}_\psi$
    \end{itemize}
  \item All nonlinear terms (blue terms in equation \ref{eq:mhdjacobian}), including perpendicular terms
    (so $\mathbb{D} = \mathbb{I}$)
\end{itemize}
This gives
\begin{eqnarray}
\mathbb{P}_{Schur} &\simeq& \mathbb{I} + \gamma^2 \frac{B_0^2}{\mu_0\rho}\left(\bvec_0\cdot\nabla\right)\delp\left(\bvec_0\cdot\nabla\right)\nabla_\perp^{-2} \nonumber \\
&\simeq& \mathbb{I} + \gamma^2 V_A^2 \left(\bvec_0\cdot\nabla\right)^2
\end{eqnarray}
Where the commutation of parallel and perpendicular derivatives is also an approximation. This
remaining term is just the shear Alfv\'en wave propagating along field-lines, the fastest wave supported by these equations.

\section{Stencils}

\section{Jacobian calculation}

The (sparse) Jacobian matrix elements can be calculated automatically
from the physics code by keeping track of the (linearised) operations
going through the RHS function.

For each point, keep the value (as usual), plus the non-zero elements in
that row of $\Jac$ and the constant: result = Ax + b
Keep track of elements using product rule.

\begin{lstlisting}
class Field3D {
  data[ngx][ngy][ngz]; // The data as now
  
  int JacIndex; // Variable index in Jacobian
  SparseMatrix *jac; // Set of rows for indices (JacIndex,*,*,*)
};
\end{lstlisting}

JacIndex is set by the solver, so for the system
\[
\Vec{f} = \left(\begin{array}{c}
P \\
\psi \\
U
\end{array}\right)
\]
\code{P.JacIndex = 0}, \code{$\psi$.JacIndex = 1} and \code{U.JacIndex = 2}.
All other fields are given \code{JacIndex = -1}. 

SparseMatrix stores the non-zero Jacobian components for the set of rows
corresponding to this variable. Evolving variables do not have an associated
\code{SparseMatrix} object, but any fields which result from operations on
evolving fields will have one.


\bibliography{references}
\bibliographystyle{unsrt}

\end{document}

