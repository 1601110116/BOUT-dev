%% Manual / derivations for preconditioning in BOUT++

\documentclass[12pt]{article}
\usepackage[nofoot]{geometry}
\usepackage{graphicx}
\usepackage{fancyhdr}
\usepackage{amsfonts}

\usepackage{listings}
\usepackage{color}
\usepackage{textcomp}
\definecolor{listinggray}{gray}{0.9}
\definecolor{lbcolor}{rgb}{0.95,0.95,0.95}
\lstset{
	backgroundcolor=\color{lbcolor},
        language=C++,
	keywordstyle=\bfseries\ttfamily\color[rgb]{0,0,1},
	identifierstyle=\ttfamily,
	commentstyle=\color[rgb]{0.133,0.545,0.133},
	stringstyle=\ttfamily\color[rgb]{0.627,0.126,0.941},
	showstringspaces=false,
	basicstyle=\small,
	numberstyle=\footnotesize,
	numbers=left,
	stepnumber=1,
	numbersep=10pt,
	tabsize=2,
	breaklines=true,
	prebreak = \raisebox{0ex}[0ex][0ex]{\ensuremath{\hookleftarrow}},
	breakatwhitespace=false,
	aboveskip={1.5\baselineskip},
        columns=fixed,
        upquote=true,
        extendedchars=true,
        morekeywords={Field2D,Field3D,Vector2D,Vector3D,real,FieldGroup},
}

%% Modify margins
\addtolength{\oddsidemargin}{-.25in}
\addtolength{\evensidemargin}{-.25in}
\addtolength{\textwidth}{0.5in}
\addtolength{\textheight}{0.25in}
%% SET HEADERS AND FOOTERS

\pagestyle{fancy}
\fancyfoot{}
\renewcommand{\sectionmark}[1]{         % Lower case Section marker style
  \markright{\thesection.\ #1}}
\fancyhead[LE,RO]{\bfseries\thepage}    % Page number (boldface) in left on even
                                        % pages and right on odd pages 
\renewcommand{\headrulewidth}{0.3pt}

\newcommand{\code}[1]{\texttt{#1}}
\newcommand{\file}[1]{\texttt{\bf #1}}

%% commands for boxes with important notes
\newlength{\notewidth}
\addtolength{\notewidth}{\textwidth}
\addtolength{\notewidth}{-3.\parindent}
\newcommand{\note}[1]{
\fbox{
\begin{minipage}{\notewidth}
{\bf NOTE}: #1
\end{minipage}
}}

\newcommand{\pow}{\ensuremath{\wedge} }

\newcommand{\deriv}[2]{\ensuremath{\frac{\partial #1}{\partial #2}}}
\newcommand{\dderiv}[2]{\ensuremath{\frac{\partial^2 #1}{\partial {#2}^2}}}
\newcommand{\Vec}[1]{\ensuremath{\mathbf{#1}}}
\newcommand{\Div}[1]{\ensuremath{\nabla\cdot #1 }}
\newcommand{\Curl}[1]{\ensuremath{\nabla\times #1 }}
\newcommand{\Bvec}{\ensuremath{\underline{B}}}
\newcommand{\bvec}{\ensuremath{\underline{b}}}
\newcommand{\kvec}{\ensuremath{\underline{\kappa}}}
\newcommand{\apar}{\ensuremath{A_{||}}}

\begin{document}

\title{DALF3 model}

\maketitle

\section{Overview}


\subsection{Normalisation}

Denote normalised quantities with a hat. 

Start with equation for total derivative:
\[
\frac{d}{dt} = \deriv{}{t} + \frac{1}{B}\bvec\times\nabla \phi \cdot\nabla
\]

Time is normalised to
\[
\hat{t} = t\frac{C_s}{L} \qquad \deriv{}{t} = \frac{C_s}{L}\deriv{}{\hat{t}}
\]
where $C_s$ is a typical sound speed $C_s = \sqrt{e\overline{T}/M_i}$, spatial derivatives to a perpendicular scale length $L$, and magnetic field to a typical magnetic field strength $\overline{B}$.

Therefore,
\[
\frac{d}{d\hat{t}} = \deriv{}{\hat{t}} + \frac{L}{C_s}\frac{1}{\overline{B}L^2\hat{B}}\bvec\times\hat{\nabla} \phi \cdot\hat{\nabla}
\]

The drift scale $\rho_s$ is given by
\[
\rho_s = \frac{C_s M_i}{e\overline{B}} \qquad \rho_sC_s = \overline{T} / \overline{B}
\]
and so
\[
\frac{d}{d\hat{t}} = \deriv{}{\hat{t}} + \frac{L}{C_s}\frac{1}{\overline{B}L^2\hat{B}}\bvec\times\hat{\nabla} \phi \cdot\hat{\nabla}
\]
Using $C_s = \overline{T} / \left(\overline{B}\rho_s\right)$,
\[
\frac{d}{d\hat{t}} = \deriv{}{\hat{t}} + \frac{\rho_s}{\overline{T}L}\frac{1}{\hat{B}}\bvec\times\hat{\nabla} \phi \cdot\hat{\nabla}
\]
Hence $\phi$ is normalised to
\[
\hat{\phi} = \frac{\phi}{\overline{T}}\frac{\rho_s}{L}
\]


\end{document}

