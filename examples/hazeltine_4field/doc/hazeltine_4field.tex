%% Documentation for BOUT++ Drift instability test

\documentclass[12pt]{article}
\usepackage[nofoot]{geometry}
\usepackage{graphicx}
\usepackage{fancyhdr}
\usepackage{subfigure}

\usepackage{listings}
\usepackage{color}
\usepackage{textcomp}
\definecolor{listinggray}{gray}{0.9}
\definecolor{lbcolor}{rgb}{0.95,0.95,0.95}
\lstset{
	backgroundcolor=\color{lbcolor},
        language=C++,
	keywordstyle=\bfseries\ttfamily\color[rgb]{0,0,1},
	identifierstyle=\ttfamily,
	commentstyle=\color[rgb]{0.133,0.545,0.133},
	stringstyle=\ttfamily\color[rgb]{0.627,0.126,0.941},
	showstringspaces=false,
	basicstyle=\small,
	numberstyle=\footnotesize,
	numbers=left,
	stepnumber=1,
	numbersep=10pt,
	tabsize=2,
	breaklines=true,
	prebreak = \raisebox{0ex}[0ex][0ex]{\ensuremath{\hookleftarrow}},
	breakatwhitespace=false,
	aboveskip={1.5\baselineskip},
        columns=fixed,
        upquote=true,
        extendedchars=true,
        morekeywords={Field2D,Field3D,Vector2D,Vector3D,real,FieldGroup},
}

%% Modify margins
\addtolength{\oddsidemargin}{-.25in}
\addtolength{\evensidemargin}{-.25in}
\addtolength{\textwidth}{0.5in}
\addtolength{\textheight}{0.25in}
%% SET HEADERS AND FOOTERS

\pagestyle{fancy}
\fancyfoot{}
\renewcommand{\sectionmark}[1]{         % Lower case Section marker style
  \markright{\thesection.\ #1}}
\fancyhead[LE,RO]{\bfseries\thepage}    % Page number (boldface) in left on even
                                        % pages and right on odd pages 
\renewcommand{\headrulewidth}{0.3pt}

%% commands for boxes with important notes
\newlength{\notewidth}
\addtolength{\notewidth}{\textwidth}
\addtolength{\notewidth}{-3.\parindent}
\newcommand{\note}[1]{
\fbox{
\begin{minipage}{\notewidth}
{\bf NOTE}: #1
\end{minipage}
}}

\newcommand{\noun}[1]{\textsc{#1}}
\newcommand{\deriv}[2]{\ensuremath{\frac{\partial #1}{\partial #2}}}
\newcommand{\code}[1]{\texttt{#1}}
\newcommand{\file}[1]{\texttt{\bf #1}}
\newcommand{\pow}{\ensuremath{\wedge}}
\newcommand{\poweq}{\ensuremath{\wedge =} }

%\newcommand{\deriv}[2]{\ensuremath{\frac{\partial #1}{\partial #2}}}
\newcommand{\apar}{\ensuremath{A_{||}}}
\newcommand{\hthe}{\ensuremath{h_\theta}}
\newcommand{\Bp}{\ensuremath{B_\theta}}
\newcommand{\Bt}{\ensuremath{B_\zeta}}
\newcommand{\Vec}[1]{\ensuremath{\mathbf{#1}}}
\newcommand{\bvec}{\Vec{b}}
\newcommand{\kvec}{\Vec{\kappa}}
\newcommand{\poisson}[2]{\ensuremath{\left[#1, #2 \right]}}
\newcommand{\bdotGxG}[2]{\ensuremath{\bvec_0\cdot\nabla_\perp #1 \times \nabla_\perp #2}}
\newcommand{\bxk}{\bvec_0\times\kvec_0\cdot\nabla}
\newcommand{\Bvec}{\Vec{B}}
\newcommand{\Bbar}{\overline{B}}
\newcommand{\Lbar}{\overline{L}}
\newcommand{\Tbar}{\overline{T}}
\newcommand{\Jvec}{\Vec{J}}
\newcommand{\Jpar}{J_{||}}
\newcommand{\delp}{\nabla_\perp^2}
\newcommand{\Div}[1]{\ensuremath{\nabla\cdot #1 }}
\newcommand{\Curl}[1]{\ensuremath{\nabla\times #1 }}
\newcommand{\rbpsq}{\ensuremath{\left(R\Bp\right)^2}}

\begin{document}

\title{Hazeltine 4-field model}
\author{B.Dudson, University of York}
\maketitle

\section{Introduction}

Implementation of the model given in ``A four-field model for tokamak plasma
dynamics'' by Hazeltine, Kotschenreuther and Morrison, Phys. Fluids {\bf 28}(8)
p2466, 1985. It evolves the pressure (through the density equation since $T$
is assumed constant), flux, vorticity and parallel velocity.
The paper uses cgs units, but here everything is given in {\bf SI units}.

{\bf Normalisation factors}: The inverse aspect ratio,
$\epsilon = a/R_0$ is assumed small ($a$ is a perpendicular scale e.g.
minor radius, $R_0$ is a reference major radius). The electron beta
$\beta_e = \mu_0n_cT_e / B_T^2$ where $n_c$ is a reference density and $T_e$
is the electron temperature in eV. The reference Alfv\'en velocity is 
$v_A = B_T / \sqrt{\mu_0 m_i n_c}$ where $m_i$ is the ion mass. This gives
a timescale $\tau_A = a / v_A$.

Two quantities $\delta$ and $\beta$ quantify the gyroradius and compressibility
effects respectively:
\begin{eqnarray*}
\delta &=& \frac{1}{2\Omega\tau_A} \qquad \Omega = \frac{eB_T}{m_i} \\
\beta &=& \frac{\beta_e}{1+\frac{1}{2}\left(1+\frac{T_i}{T_e}\right)\beta_e} 
\end{eqnarray*}

Time is normalised to $\tau_A / \epsilon$ and evolving quantities are 
normalised as follows:
\begin{eqnarray*}
\tau &=& \epsilon t / \tau_A \\
p &=& \frac{\beta_e}{\epsilon}\left(\frac{n}{n_c} - 1\right) \\
v &=& \frac{1}{\epsilon v_A}v_{||} \\
\varphi &=& \frac{\phi}{\epsilon v_A B_T a}
\end{eqnarray*}
Resistivity is normalised as $\eta = \tau_A/\left(\epsilon \tau_S\right)$
where $\tau_S$ is the skin time.

Poisson brackets are used for perpendicular advection terms,
and $\nabla_{||}$ is a derivative along the perturbed magnetic field:
\begin{eqnarray*}
\left[f, g\right] &=& \bdotGxG{f}{g} \\
\nabla_{||}f &=& \bvec_0\cdot\nabla f - \left[\psi, f\right]
\end{eqnarray*}

\subsection{Vorticity equation}

\[
\deriv{U}{\tau} + \left[\varphi + \delta\frac{T_i}{T_e}p, U\right] + \nabla_{||}J + \left(1+\frac{T_i}{T_e}\right)\left[h, p\right] = \delta\frac{T_i}{T_e}\left[\nabla_\perp\varphi ; \nabla_\perp p\right]
\]

The quantity $h \equiv \left(R - R_0\right)/a$ measures the curvature of the 
equilibrium magnetic field. 

\[
U = \nabla_\perp^2 \varphi
\]

\subsection{Parallel electric field}

\[
\deriv{\psi}{\tau} + \nabla_{||}\varphi = \eta J + \delta\nabla_{||} p
\]

\[
J = \nabla_\perp^2 \psi
\]

\subsection{Parallel velocity}

\[
\deriv{v}{\tau} + \poisson{\varphi}{v} + \frac{1}{2}\left(1+\frac{T_i}{T_e}\right)\nabla_{||}p = \delta\beta\frac{T_i}{T_e}\left\{\frac{1}{2}\left(1+\frac{T_i}{T_e}\right)\poisson{p}{v} + 4\poisson{h}{v}\right\}
\]

\subsection{Pressure}

Evolving the density equation since the temperature is assumed constant.

\[
\deriv{p}{\tau} + \poisson{\varphi}{p} = \beta\left\{ 2\poisson{h}{\varphi - \delta p} - \nabla_{||}\left(v + 2\delta J\right) + \frac{1}{2}\left(1+\frac{T_i}{T_e}\right)\eta\nabla_\perp^2p\right\}
\]

\section{Diagnostics}

\subsection{Energy}

The four-field energy $H$ is conserved if $\eta = 0$
\[
H = \frac{1}{2}\int dV \left[ \left|\nabla_\perp\varphi\right|^2 + \left|\nabla_\perp\psi\right|^2 + \frac{p^2}{2\beta} - 2\frac{T_i}{T_e}hp - \frac{\psi v}{2\delta} + \left(\frac{1}{2\left(1+T_i/T_e\right)} + \frac{T_i\beta}{4T_e}\right)v^2\right]
\]

\end{document}
